\documentclass{article}
\usepackage{ctex}

\title{\songti 数字逻辑与计算机组成\\ {\small 实验 1: 基本逻辑部件设计}}
\author{\kaishu 姓名\quad 学号}
\date{\zhtoday}

\usepackage{hyperref}
\usepackage{algorithm}
\usepackage{algorithmicx}
\usepackage{algpseudocode}
\usepackage{float}  
\usepackage{lipsum}
\usepackage{color, xcolor}
\usepackage{listings}
\usepackage{dirtree}
\usepackage{ulem}
\usepackage{graphicx}
\usepackage{amsmath}
\usepackage{amssymb}
\usepackage{amsfonts}
\usepackage{xcolor}
\usepackage{tikz}
\usepackage{zhnumber} % change section number to chinese
\renewcommand\thesection{\zhnum{section}}
\renewcommand \thesubsection {\arabic{section}}
\usetikzlibrary{arrows,shapes,chains}

\begin{document}
    \maketitle

    \kaishu

    \section{实验目的}

    \begin{enumerate}
        \item 熟悉 Logisim 软件的使用方法。
        \item 掌握使用晶体管实现基本逻辑部件的方法。
        \item 利用基础元器件库设计简单数字电路。
        \item 掌握子电路的设计和应用。
        \item 掌握分线器、隧道、探针等 Logisim 组件的使用方法。
    \end{enumerate}

    \section{实验环境}

    Logisim:https://github.com/Logisim-Ita/Logisim

    \section{实验内容}


\end{document}